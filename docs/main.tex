\documentclass[oneside, final, 14pt]{extarticle}

\usepackage[utf8]{inputenc}
\usepackage[russianb]{babel}
\usepackage{vmargin}
\usepackage{verbatim}
\setpapersize{A4}
\setmarginsrb{3cm}{2cm}{1cm}{2cm}{0pt}{0mm}{0pt}{13mm}
\usepackage{indentfirst}
\usepackage{graphicx}
\sloppy


\begin{document}

\thispagestyle{empty}

\begin{center}
\ \vspace{-3cm}

\includegraphics[width=0.5\textwidth]{msu.eps}\\
{\scshape Московский государственный университет имени М.~В.~Ломоносова}\\
Факультет вычислительной математики и кибернетики\\
Кафедра алгоритмических языков

\vfill

\vspace{7mm}

{\Large \bfseries \texttt{ImageBoard}\\}

\vspace{5mm}

\end{center}

\vspace{5cm}

\begin{flushright}
Кирилл Алексеев\\
Группа 524
\end{flushright}

\vfill

\centerline {Москва, 2017}

\clearpage

\begingroup
\let\clearpage\relax
\tableofcontents
\endgroup

\clearpage

\section{Функциональные требования}

Необходимо реализовать веб-приложение типа \texttt{imageboard} с единственной темой на главной странице. Пользователи делятся на три основные группы: обычные пользователи, администраторы и суперпользователи. Функциональные требования к реализуемому сервису можно разделить на несколько категорий:

\begin{itemize}
\item{основная функциональность;}
\item{профили пользователей;}
\item{интерфейс администратора.}
\end{itemize}

\vspace{0.5cm}

Основная функциональность:
\begin{itemize}
\item{сообщения могут оставлять как зарегистрированные, так и анонимные пользователи;}
\item{зарегистрированные пользователи могу редактировать и удалять свои сообщения;}
\item{сообщения имеют дату, автора, тему и тело;}
\item{тело сообщения может содержать изображения, загружаемые на сервер;}
\item{на главную страницу выводится только N последних сообщений, все остальные на следующих страницах не более, чем по N штук на каждой.}
\end{itemize}

\vspace{0.5cm}

Требования, предъявляемые к профилям пользователей:
\begin{itemize}
\item{новый пользователь может зарегистрироваться по логину, паролю и email;}
\item{при регистрации пользователь должен подтвердить свой email;}
\item{пользователи аутентифицируются по логину и паролю;}
\item{пользователь может разлогиниваться;}
\item{пользователь может загружать изображения профиля, менять пароль и email.}
\end{itemize}

\vspace{0.5cm}

Требования, предъявляемые к интерфейсу администратора:
\begin{itemize}
\item{администратор может удалять любые сообщения;}
\item{администратор может менять пароль обычному пользователю;}
\item{администратор может добавить произвольного пользователя в группу администраторов.}
\end{itemize}


\section{Требования информационной безопасности}

Требования информационной безопасности можно разделить на несколько подкатегорий.

\vspace{0.5cm}

Все пользователи:
\begin{itemize}
\item{никто не может менять чужие данные профиля (кроме администраторов);}
\item{никто не может менять чужой пароль (кроме администраторов);}
\item{никто не может удалять чужие сообщения (кроме администраторов);}
\item{никто не может редактировать чужие сообщения;}
\end{itemize}

\vspace{0.5cm}

Приложение:
\begin{itemize}
\item{нельзя хранить пароли в открытом виде.}
\end{itemize}

\vspace{0.5cm}

Эксплуатация:
\begin{itemize}
\item{должно быть ограничение на размер изображения;}
\item{должно быть ограничение на количество запросов за определенный временной период;}
\item{связь с внешним окружением только через порты 80 и 443;}
\end{itemize}

\vspace{0.5cm}

Администратор:
\begin{itemize}
\item{администратор не может менять пароль другому администратору.}
\end{itemize}

\section{Модель угроз}
\subsection{Матрица доступов}

\begin{tabular}[width=\textwidth]{| l | l | l | l | l |}
\hline
& Аноним & Обычный пользователь & Администратор & Суперпользователь \\
\hline
Профиль пользователя &&&& \\ \hline
Пароль пользователя &&&& \\ \hline
Сообщение пользователя &&&& \\ \hline
\end{tabular}

\subsection{Потоки данных}
\includegraphics{dataflow.eps}\\

\subsection{Деревья угроз}

\vspace{1cm}
\noindent
\includegraphics[width=\textwidth]{ddos.eps}\\
\vspace{1cm}\\
\includegraphics[width=\textwidth]{pass.eps}\\
\vspace{1cm}\\
\includegraphics[width=\textwidth]{delete.eps}\\

\section{Обоснование выбранных технологий} 

При разработке приложения были использованы следующие инструменты:

\begin{itemize}
\item{\texttt{python3};}
\item{\texttt{django};}
\item{\texttt{gunicorn};}
\item{\texttt{nginx};}
\item{\texttt{docker};}
\item{\texttt{docker-compose};}
\item{\texttt{git}.}
\end{itemize}

\vspace{0.5cm}

Для всех выбранных технологий справедливо:

\begin{itemize}
\item{технология хорошо протестирована и используется многими известными вендорами;}
\item{технология хорошо поддерживается сообществом;}
\item{у разработчика есть экспертиза в использовании технологии (кроме \texttt{Docker}).}
\end{itemize}

\vspace{0.5cm}

Основным инструментом, влияющим на выбор остальных, является web-framework \texttt{Django},
написанный на языке программирования \texttt{Python}. Использование \texttt{Django} влечет за собой
использование языка программирования \texttt{Python} и веб-сервера  \texttt{Gunicorn}, который
используется для развертывания приложений, написанных на \texttt{Django}.

Выбор фреймворка \texttt{Django} обоснован следующими факторами:

\begin{itemize}
\item{\texttt{Django} позволяет легко разрабатывать веб-приложения;}
\item{\texttt{Django} имеет ORM;}
\item{\texttt{Django} позволяет легко разработать интерфейс администратора;}
\end{itemize}

\vspace{0.5cm}

В качестве frontend веб-сервера был выбран \texttt{nginx} для следующих задач:

\begin{itemize}
\item{фильтрация и валидация входящих запросов;}
\item{осблуживание медленных клиентов;}
\item{обслуживание запросов на статические файлы;}
\item{поддержка \texttt{HTTPS};}
\item{проксирование запросов в \texttt{Gunicorn}.}
\end{itemize}

\vspace{0.5cm}

Для виртуализации и развертывания был выбран \texttt{Docker} по следующим причинам:

\begin{itemize}
\item{\texttt{Docker} позволяет легко изолировать отдельное приложение;}
\item{\texttt{Docker} позволяет легко развернуть сервис на любой популярной ОС;}
\item{\texttt{Docker} повзоляет легко настроить сетевое взаимодействие между отдельными компонентами сервиса;}
\item{разработчик сервиса хотел научиться пользоваться этим инструментом.}
\end{itemize}

\section{Соглашение о кодировании}

Для кодирования были приняты следующие соглашения:

\begin{itemize}
\item{по максимуму использовать возможности \texttt{Django}, делать велосипеды только если в \texttt{Django} отсутствует [адекватная] возможность реализации нужной функциональности;}
\item{доработка функциональности авторизации проводится путем наследования от стандартных моделей \texttt{Django};}
\item{использовать ORM, не писать \texttt{raw-SQL} запросы;}
\item{все методы \texttt{API} начинаются на \texttt{/api/}.}
\end{itemize}

\section{Сценарии ручного тестирования}

Тестирование основной функциональности для анонимного пользователя:

\begin{enumerate}
\item{Зайти на главную страницу.}
\item{Заполнить поля \texttt{Subject} и \texttt{Body}, приложить изображение, нажать на кнопку \texttt{Publish}.}
\item{Увидеть опубликованное сообщение с автором \texttt{anonymous}.}
\end{enumerate}

\vspace{0.5cm}

Тестирование регистрации нового пользователя:

\begin{enumerate}
\item{Зайти на главную страницу.}
\item{Нажать на ссылку \texttt{Sign Up}.}
\item{Заполнить поля \texttt{Username}, \texttt{Email} и \texttt{Password}, нажать на кнопку \texttt{Sign Up}.}
\item{Пройти по ссылке в письме, отправленном на указанный \texttt{email}, увидеть приветствие на главной странице.}
\end{enumerate}

\vspace{0.5cm}

Тестирование основной функциональности для зарегистрированного пользователя:

\begin{enumerate}
\item{Зайти на главную страницу.}
\item{Нажать на ссылку \texttt{Sign In}, авторизоваться.}
\item{Заполнить поля \texttt{Subject} и \texttt{Body}, приложить изображение, нажать на кнопку \texttt{Publish}.}
\item{Нажать на ссылку \texttt{edit} возле опубликованного сообщения, поменять произвольные поля, нажать на кнопку \texttt{Save}.}
\item{Убедиться, что содержание сообщения изменилось.}
\item{Нажать на ссылку \texttt{delete}, убедиться, что сообщение пропало.}
\end{enumerate}

\vspace{0.5cm}

Тестирование редактирования профиля:

\begin{enumerate}
\item{Зайти на главную страницу.}
\item{Нажать на ссылку \texttt{Sign In}, авторизоваться.}
\item{Нажать на ссылку \texttt{Profile}.}
\item{Поменять произвольные поля, нажать на кнопку \texttt{Save}.}
\end{enumerate}

\vspace{0.5cm}

Тестирование смены пароля:

\begin{enumerate}
\item{Зайти на главную страницу.}
\item{Нажать на ссылку \texttt{Sign In}, авторизоваться.}
\item{Нажать на ссылку \texttt{Change Password}.}
\item{Ввести новый пароль, нажать на кнопку \texttt{Save}.}
\item{Нажать на ссылку \texttt{Sign Out}.}
\item{Нажать на ссылку \texttt{Sign In}, авторизоваться по новому паролю.}
\end{enumerate}

\vspace{0.5cm}

Тестирование пагинации:

\begin{enumerate}
\item{Зайти на главную страницу.}
\item{Создать более N сообщений (по умолчанию 5).}
\item{Убедиться, что одновременно отображается не более 5 сообщений.}
\item{Перейти на следующую страницу по ссылке внизу страницы.}
\item{Увидеть оставшиеся сообщения.}
\end{enumerate}

\vspace{0.5cm}

Тестирование повышения привилегий:

\begin{enumerate}
\item{Зайти на главную страницу.}
\item{Зарегистрировать и активировать нового пользователя.}
\item{Разлогиниться, залогиниться под суперюзером.}
\item{Нажать на ссылку \texttt{Admin}.}
\item{Нажать на ссылку \texttt{Groups}.}
\item{Нажать на кнопку \texttt{Add Group}.}
\item{В поле \texttt{Name} ввести \texttt{admins}, добавить разрешения "Can delete message" и "Can change user", нажать на кнопку \texttt{Save}.}
\item{Нажать на ссылку \texttt{Home}.}
\item{Нажать на ссылку \texttt{Users}.}
\item{Выбрать зарегистрированного пользователь без повышенных привилегий.}
\item{Отметить поле "Staff status", добавить пользователя в группу \texttt{admins}, нажать на кнопку \texttt{Save}.}
\end{enumerate}

\vspace{0.5cm}

Тестирование удаления сообщений администратором:

\begin{enumerate}
\item{Зайти на главную страницу.}
\item{Создать произвольное количество сообщений.}
\item{Залогиниться под пользователем из группы администраторов.}
\item{Нажать на ссылку \texttt{Admin}.}
\item{Нажать на ссылку \texttt{Messages}.}
\item{Выбрать произвольное количество сообщений, в выпадающем меню выбрать \texttt{Delete selected messages}, нажать на кнопку \texttt{Go}, подтвердить действие.}
\texttt{Убедиться, что выбранные сообщения были удалены.}
\end{enumerate}

\vspace{0.5cm}

Тестирование смены пароля обычному пользователю администратором:

\begin{enumerate}
\item{Зайти на главную страницу.}
\item{Зарегистрировать и активировать нового пользователя.}
\item{Залогиниться под пользователем из группы администраторов.}
\item{Нажать на ссылку \texttt{Admin}.}
\item{Нажать на ссылку \texttt{Users}.}
\item{Выбрать пользователя без повышенных привилегий.}
\item{Нажать на ссылку \texttt{this form}.}
\item{Ввести новый пароль, нажать на кнопку \texttt{Change Password}.}
\item{Выбрать пользователя с повышенными привилегиями, получить 403.}
\item{Разлогиниться, залогиниться под пользователем с обычными привилегиями с новым паролем.}
\end{enumerate}

\end{document}

